%% Generated by Sphinx.
\def\sphinxdocclass{report}
\documentclass[letterpaper,10pt,english]{sphinxmanual}
\ifdefined\pdfpxdimen
   \let\sphinxpxdimen\pdfpxdimen\else\newdimen\sphinxpxdimen
\fi \sphinxpxdimen=.75bp\relax

\PassOptionsToPackage{warn}{textcomp}
\usepackage[utf8]{inputenc}
\ifdefined\DeclareUnicodeCharacter
% support both utf8 and utf8x syntaxes
  \ifdefined\DeclareUnicodeCharacterAsOptional
    \def\sphinxDUC#1{\DeclareUnicodeCharacter{"#1}}
  \else
    \let\sphinxDUC\DeclareUnicodeCharacter
  \fi
  \sphinxDUC{00A0}{\nobreakspace}
  \sphinxDUC{2500}{\sphinxunichar{2500}}
  \sphinxDUC{2502}{\sphinxunichar{2502}}
  \sphinxDUC{2514}{\sphinxunichar{2514}}
  \sphinxDUC{251C}{\sphinxunichar{251C}}
  \sphinxDUC{2572}{\textbackslash}
\fi
\usepackage{cmap}
\usepackage[T1]{fontenc}
\usepackage{amsmath,amssymb,amstext}
\usepackage{babel}



\usepackage{times}
\expandafter\ifx\csname T@LGR\endcsname\relax
\else
% LGR was declared as font encoding
  \substitutefont{LGR}{\rmdefault}{cmr}
  \substitutefont{LGR}{\sfdefault}{cmss}
  \substitutefont{LGR}{\ttdefault}{cmtt}
\fi
\expandafter\ifx\csname T@X2\endcsname\relax
  \expandafter\ifx\csname T@T2A\endcsname\relax
  \else
  % T2A was declared as font encoding
    \substitutefont{T2A}{\rmdefault}{cmr}
    \substitutefont{T2A}{\sfdefault}{cmss}
    \substitutefont{T2A}{\ttdefault}{cmtt}
  \fi
\else
% X2 was declared as font encoding
  \substitutefont{X2}{\rmdefault}{cmr}
  \substitutefont{X2}{\sfdefault}{cmss}
  \substitutefont{X2}{\ttdefault}{cmtt}
\fi


\usepackage[Bjarne]{fncychap}
\usepackage{sphinx}

\fvset{fontsize=\small}
\usepackage{geometry}

% Include hyperref last.
\usepackage{hyperref}
% Fix anchor placement for figures with captions.
\usepackage{hypcap}% it must be loaded after hyperref.
% Set up styles of URL: it should be placed after hyperref.
\urlstyle{same}
\addto\captionsenglish{\renewcommand{\contentsname}{Contents:}}

\usepackage{sphinxmessages}
\setcounter{tocdepth}{1}



\title{pytexexam}
\date{Jan 08, 2020}
\release{1.3}
\author{Vu Ngoc Binh}
\newcommand{\sphinxlogo}{\vbox{}}
\renewcommand{\releasename}{Release}
\makeindex
\begin{document}

\pagestyle{empty}
\sphinxmaketitle
\pagestyle{plain}
\sphinxtableofcontents
\pagestyle{normal}
\phantomsection\label{\detokenize{index::doc}}



\chapter{Pytexexam main class}
\label{\detokenize{index:pytexexam-main-class}}\index{Answer (class in pytexexam.answer)@\spxentry{Answer}\spxextra{class in pytexexam.answer}}

\begin{fulllineitems}
\phantomsection\label{\detokenize{index:pytexexam.answer.Answer}}\pysiglinewithargsret{\sphinxbfcode{\sphinxupquote{class }}\sphinxcode{\sphinxupquote{pytexexam.answer.}}\sphinxbfcode{\sphinxupquote{Answer}}}{\emph{answer: str = ''}, \emph{is\_true\_answer=False}}{}
This class is used to store 1 answer in a exam question.

\end{fulllineitems}

\phantomsection\label{\detokenize{index:module-pytexexam.question}}\index{pytexexam.question (module)@\spxentry{pytexexam.question}\spxextra{module}}\index{Question (class in pytexexam.question)@\spxentry{Question}\spxextra{class in pytexexam.question}}

\begin{fulllineitems}
\phantomsection\label{\detokenize{index:pytexexam.question.Question}}\pysiglinewithargsret{\sphinxbfcode{\sphinxupquote{class }}\sphinxcode{\sphinxupquote{pytexexam.question.}}\sphinxbfcode{\sphinxupquote{Question}}}{\emph{question: str}}{}
This class represents one question on the test.
\index{answer\_a() (pytexexam.question.Question method)@\spxentry{answer\_a()}\spxextra{pytexexam.question.Question method}}

\begin{fulllineitems}
\phantomsection\label{\detokenize{index:pytexexam.question.Question.answer_a}}\pysiglinewithargsret{\sphinxbfcode{\sphinxupquote{answer\_a}}}{\emph{answer: str}, \emph{true\_answer=False}}{}
This method is used to enter answer A for the question.
\begin{quote}\begin{description}
\item[{Parameters}] \leavevmode\begin{itemize}
\item {} 
\sphinxstyleliteralstrong{\sphinxupquote{answer}} \textendash{} Content of the answer A

\item {} 
\sphinxstyleliteralstrong{\sphinxupquote{true\_answer}} \textendash{} If this is the correct answer then enter True. otherwise False

\end{itemize}

\end{description}\end{quote}

\end{fulllineitems}

\index{answer\_b() (pytexexam.question.Question method)@\spxentry{answer\_b()}\spxextra{pytexexam.question.Question method}}

\begin{fulllineitems}
\phantomsection\label{\detokenize{index:pytexexam.question.Question.answer_b}}\pysiglinewithargsret{\sphinxbfcode{\sphinxupquote{answer\_b}}}{\emph{answer: str}, \emph{true\_answer=False}}{}
This method is used to enter answer B to the question.
\begin{quote}\begin{description}
\item[{Parameters}] \leavevmode\begin{itemize}
\item {} 
\sphinxstyleliteralstrong{\sphinxupquote{answer}} \textendash{} Content of the answer B

\item {} 
\sphinxstyleliteralstrong{\sphinxupquote{true\_answer}} \textendash{} If this is the correct answer then enter True, otherwise False

\end{itemize}

\end{description}\end{quote}

\end{fulllineitems}

\index{answer\_c() (pytexexam.question.Question method)@\spxentry{answer\_c()}\spxextra{pytexexam.question.Question method}}

\begin{fulllineitems}
\phantomsection\label{\detokenize{index:pytexexam.question.Question.answer_c}}\pysiglinewithargsret{\sphinxbfcode{\sphinxupquote{answer\_c}}}{\emph{answer: str}, \emph{true\_answer=False}}{}
This method is used to enter answer C to the question.
\begin{quote}\begin{description}
\item[{Parameters}] \leavevmode\begin{itemize}
\item {} 
\sphinxstyleliteralstrong{\sphinxupquote{answer}} \textendash{} Content of the answer C

\item {} 
\sphinxstyleliteralstrong{\sphinxupquote{true\_answer}} \textendash{} If this is the correct answer then enter True, otherwise False

\end{itemize}

\end{description}\end{quote}

\end{fulllineitems}

\index{answer\_d() (pytexexam.question.Question method)@\spxentry{answer\_d()}\spxextra{pytexexam.question.Question method}}

\begin{fulllineitems}
\phantomsection\label{\detokenize{index:pytexexam.question.Question.answer_d}}\pysiglinewithargsret{\sphinxbfcode{\sphinxupquote{answer\_d}}}{\emph{answer: str}, \emph{true\_answer=False}}{}
This method is used to enter answer D for the question.
\begin{quote}\begin{description}
\item[{Parameters}] \leavevmode\begin{itemize}
\item {} 
\sphinxstyleliteralstrong{\sphinxupquote{answer}} \textendash{} Content of the answer D

\item {} 
\sphinxstyleliteralstrong{\sphinxupquote{true\_answer}} \textendash{} If this is the correct answer then enter True, otherwise False

\end{itemize}

\end{description}\end{quote}

\end{fulllineitems}

\index{answers() (pytexexam.question.Question method)@\spxentry{answers()}\spxextra{pytexexam.question.Question method}}

\begin{fulllineitems}
\phantomsection\label{\detokenize{index:pytexexam.question.Question.answers}}\pysiglinewithargsret{\sphinxbfcode{\sphinxupquote{answers}}}{\emph{true\_answer: str, answer\_dict: Dict{[}str, str{]}}}{}
Another way to enter answers to questions.
\begin{quote}\begin{description}
\item[{Parameters}] \leavevmode\begin{itemize}
\item {} 
\sphinxstyleliteralstrong{\sphinxupquote{true\_answer}} \textendash{} The letter that corresponds to the correct answer (A, B, C, D)

\item {} 
\sphinxstyleliteralstrong{\sphinxupquote{answer\_dict}} \textendash{} A dictionary contains the answers to the questions.         The corresponding key of this dictionary is A, B, C, D.

\end{itemize}

\end{description}\end{quote}

\end{fulllineitems}

\index{get\_answer() (pytexexam.question.Question method)@\spxentry{get\_answer()}\spxextra{pytexexam.question.Question method}}

\begin{fulllineitems}
\phantomsection\label{\detokenize{index:pytexexam.question.Question.get_answer}}\pysiglinewithargsret{\sphinxbfcode{\sphinxupquote{get\_answer}}}{\emph{answer\_key: str}}{{ $\rightarrow$ str}}
This method is used to get answers to questions.
\begin{quote}\begin{description}
\item[{Parameters}] \leavevmode
\sphinxstyleliteralstrong{\sphinxupquote{answer\_key}} \textendash{} The key corresponding to the answer of the question.

\item[{Returns}] \leavevmode
The answer corresponds to the selected answer.

\end{description}\end{quote}

\end{fulllineitems}

\index{get\_answer\_column() (pytexexam.question.Question method)@\spxentry{get\_answer\_column()}\spxextra{pytexexam.question.Question method}}

\begin{fulllineitems}
\phantomsection\label{\detokenize{index:pytexexam.question.Question.get_answer_column}}\pysiglinewithargsret{\sphinxbfcode{\sphinxupquote{get\_answer\_column}}}{}{{ $\rightarrow$ int}}
This method returns the number of columns where the answer will be presented when the
question is printed. The function can return 1, 2, 4.
\begin{quote}\begin{description}
\item[{Returns}] \leavevmode
The number of columns the answer will be displayed when the question is printed

\end{description}\end{quote}

\end{fulllineitems}

\index{get\_true\_answer() (pytexexam.question.Question method)@\spxentry{get\_true\_answer()}\spxextra{pytexexam.question.Question method}}

\begin{fulllineitems}
\phantomsection\label{\detokenize{index:pytexexam.question.Question.get_true_answer}}\pysiglinewithargsret{\sphinxbfcode{\sphinxupquote{get\_true\_answer}}}{}{{ $\rightarrow$ str}}
This method returns the character corresponding to the correct answer of the question.
The possible answer are A, B, C, D.
\begin{quote}\begin{description}
\item[{Returns}] \leavevmode
The letter corresponding to the correct answer of the question

\end{description}\end{quote}

\end{fulllineitems}

\index{question (pytexexam.question.Question attribute)@\spxentry{question}\spxextra{pytexexam.question.Question attribute}}

\begin{fulllineitems}
\phantomsection\label{\detokenize{index:pytexexam.question.Question.question}}\pysigline{\sphinxbfcode{\sphinxupquote{question}}\sphinxbfcode{\sphinxupquote{ = None}}}
Content of the question.

\end{fulllineitems}

\index{set\_answer\_column() (pytexexam.question.Question method)@\spxentry{set\_answer\_column()}\spxextra{pytexexam.question.Question method}}

\begin{fulllineitems}
\phantomsection\label{\detokenize{index:pytexexam.question.Question.set_answer_column}}\pysiglinewithargsret{\sphinxbfcode{\sphinxupquote{set\_answer\_column}}}{\emph{answer\_column: int}}{}
This method allows you to enter the number of columns where the answer will be displayed
when printing the question. The possible values ​​are 1, 2, 4
\begin{quote}\begin{description}
\item[{Parameters}] \leavevmode
\sphinxstyleliteralstrong{\sphinxupquote{answer\_column}} \textendash{} The number of columns the answer will be displayed when printed.

\end{description}\end{quote}

\end{fulllineitems}

\index{shuffle\_answer() (pytexexam.question.Question method)@\spxentry{shuffle\_answer()}\spxextra{pytexexam.question.Question method}}

\begin{fulllineitems}
\phantomsection\label{\detokenize{index:pytexexam.question.Question.shuffle_answer}}\pysiglinewithargsret{\sphinxbfcode{\sphinxupquote{shuffle\_answer}}}{}{}
The method that allows the swap answers in question.

\end{fulllineitems}

\index{solution() (pytexexam.question.Question method)@\spxentry{solution()}\spxextra{pytexexam.question.Question method}}

\begin{fulllineitems}
\phantomsection\label{\detokenize{index:pytexexam.question.Question.solution}}\pysiglinewithargsret{\sphinxbfcode{\sphinxupquote{solution}}}{\emph{solution: str}}{}
This method is used to enter detailed answer to the question

\end{fulllineitems}


\end{fulllineitems}

\phantomsection\label{\detokenize{index:module-pytexexam.exam}}\index{pytexexam.exam (module)@\spxentry{pytexexam.exam}\spxextra{module}}\index{Exam (class in pytexexam.exam)@\spxentry{Exam}\spxextra{class in pytexexam.exam}}

\begin{fulllineitems}
\phantomsection\label{\detokenize{index:pytexexam.exam.Exam}}\pysiglinewithargsret{\sphinxbfcode{\sphinxupquote{class }}\sphinxcode{\sphinxupquote{pytexexam.exam.}}\sphinxbfcode{\sphinxupquote{Exam}}}{\emph{question\_list: List{[}question.Question{]}}}{}
This class represents an exam.
\index{question\_list (pytexexam.exam.Exam attribute)@\spxentry{question\_list}\spxextra{pytexexam.exam.Exam attribute}}

\begin{fulllineitems}
\phantomsection\label{\detokenize{index:pytexexam.exam.Exam.question_list}}\pysigline{\sphinxbfcode{\sphinxupquote{question\_list}}\sphinxbfcode{\sphinxupquote{ = None}}}
List of questions in the exam

\end{fulllineitems}

\index{shuffle\_question() (pytexexam.exam.Exam method)@\spxentry{shuffle\_question()}\spxextra{pytexexam.exam.Exam method}}

\begin{fulllineitems}
\phantomsection\label{\detokenize{index:pytexexam.exam.Exam.shuffle_question}}\pysiglinewithargsret{\sphinxbfcode{\sphinxupquote{shuffle\_question}}}{}{}
This method allows to shuffle all the questions in the exam.

\end{fulllineitems}


\end{fulllineitems}

\phantomsection\label{\detokenize{index:module-pytexexam.latexexam}}\index{pytexexam.latexexam (module)@\spxentry{pytexexam.latexexam}\spxextra{module}}\index{LatexExam (class in pytexexam.latexexam)@\spxentry{LatexExam}\spxextra{class in pytexexam.latexexam}}

\begin{fulllineitems}
\phantomsection\label{\detokenize{index:pytexexam.latexexam.LatexExam}}\pysiglinewithargsret{\sphinxbfcode{\sphinxupquote{class }}\sphinxcode{\sphinxupquote{pytexexam.latexexam.}}\sphinxbfcode{\sphinxupquote{LatexExam}}}{\emph{exam\_title: str}, \emph{exam: exam.Exam}}{}
This class represents a exam, allowing users to print the exam and answer to a tex file
or pdf (with latex pre-installed)
\index{add\_user\_preamble() (pytexexam.latexexam.LatexExam method)@\spxentry{add\_user\_preamble()}\spxextra{pytexexam.latexexam.LatexExam method}}

\begin{fulllineitems}
\phantomsection\label{\detokenize{index:pytexexam.latexexam.LatexExam.add_user_preamble}}\pysiglinewithargsret{\sphinxbfcode{\sphinxupquote{add\_user\_preamble}}}{\emph{preamble: str}}{}
Added preamble of latex file

\end{fulllineitems}

\index{exam\_content (pytexexam.latexexam.LatexExam attribute)@\spxentry{exam\_content}\spxextra{pytexexam.latexexam.LatexExam attribute}}

\begin{fulllineitems}
\phantomsection\label{\detokenize{index:pytexexam.latexexam.LatexExam.exam_content}}\pysigline{\sphinxbfcode{\sphinxupquote{exam\_content}}\sphinxbfcode{\sphinxupquote{ = None}}}
The content of the exam

\end{fulllineitems}

\index{exam\_header (pytexexam.latexexam.LatexExam attribute)@\spxentry{exam\_header}\spxextra{pytexexam.latexexam.LatexExam attribute}}

\begin{fulllineitems}
\phantomsection\label{\detokenize{index:pytexexam.latexexam.LatexExam.exam_header}}\pysigline{\sphinxbfcode{\sphinxupquote{exam\_header}}\sphinxbfcode{\sphinxupquote{ = None}}}
The presentation of the exam’s header

\end{fulllineitems}

\index{exam\_title (pytexexam.latexexam.LatexExam attribute)@\spxentry{exam\_title}\spxextra{pytexexam.latexexam.LatexExam attribute}}

\begin{fulllineitems}
\phantomsection\label{\detokenize{index:pytexexam.latexexam.LatexExam.exam_title}}\pysigline{\sphinxbfcode{\sphinxupquote{exam\_title}}\sphinxbfcode{\sphinxupquote{ = None}}}
Exam name

\end{fulllineitems}

\index{export\_pdf\_answer() (pytexexam.latexexam.LatexExam method)@\spxentry{export\_pdf\_answer()}\spxextra{pytexexam.latexexam.LatexExam method}}

\begin{fulllineitems}
\phantomsection\label{\detokenize{index:pytexexam.latexexam.LatexExam.export_pdf_answer}}\pysiglinewithargsret{\sphinxbfcode{\sphinxupquote{export\_pdf\_answer}}}{\emph{file\_name: str}}{}
This method export the answer as a tex file.
\begin{quote}\begin{description}
\item[{Parameters}] \leavevmode
\sphinxstyleliteralstrong{\sphinxupquote{file\_name}} \textendash{} The file name will output.

\end{description}\end{quote}

\end{fulllineitems}

\index{export\_pdf\_exam() (pytexexam.latexexam.LatexExam method)@\spxentry{export\_pdf\_exam()}\spxextra{pytexexam.latexexam.LatexExam method}}

\begin{fulllineitems}
\phantomsection\label{\detokenize{index:pytexexam.latexexam.LatexExam.export_pdf_exam}}\pysiglinewithargsret{\sphinxbfcode{\sphinxupquote{export\_pdf\_exam}}}{\emph{file\_name: str}}{}
This method export the exam as a pdf file.
\begin{quote}\begin{description}
\item[{Parameters}] \leavevmode
\sphinxstyleliteralstrong{\sphinxupquote{file\_name}} \textendash{} The file name will output.

\end{description}\end{quote}

\end{fulllineitems}

\index{export\_pdf\_solution() (pytexexam.latexexam.LatexExam method)@\spxentry{export\_pdf\_solution()}\spxextra{pytexexam.latexexam.LatexExam method}}

\begin{fulllineitems}
\phantomsection\label{\detokenize{index:pytexexam.latexexam.LatexExam.export_pdf_solution}}\pysiglinewithargsret{\sphinxbfcode{\sphinxupquote{export\_pdf\_solution}}}{\emph{file\_name: str}}{}
Export a file containing detailed answers for each question in the exam

\end{fulllineitems}

\index{export\_tex\_answer() (pytexexam.latexexam.LatexExam method)@\spxentry{export\_tex\_answer()}\spxextra{pytexexam.latexexam.LatexExam method}}

\begin{fulllineitems}
\phantomsection\label{\detokenize{index:pytexexam.latexexam.LatexExam.export_tex_answer}}\pysiglinewithargsret{\sphinxbfcode{\sphinxupquote{export\_tex\_answer}}}{\emph{file\_name: str}}{}
This method export the answer as a tex file.
\begin{quote}\begin{description}
\item[{Parameters}] \leavevmode
\sphinxstyleliteralstrong{\sphinxupquote{file\_name}} \textendash{} The file name will output.

\end{description}\end{quote}

\end{fulllineitems}

\index{export\_tex\_exam() (pytexexam.latexexam.LatexExam method)@\spxentry{export\_tex\_exam()}\spxextra{pytexexam.latexexam.LatexExam method}}

\begin{fulllineitems}
\phantomsection\label{\detokenize{index:pytexexam.latexexam.LatexExam.export_tex_exam}}\pysiglinewithargsret{\sphinxbfcode{\sphinxupquote{export\_tex\_exam}}}{\emph{file\_name: str}}{}
This method proposed exam as a tex file.
\begin{quote}\begin{description}
\item[{Parameters}] \leavevmode
\sphinxstyleliteralstrong{\sphinxupquote{file\_name}} \textendash{} The file name will output.

\end{description}\end{quote}

\end{fulllineitems}

\index{export\_tex\_solution() (pytexexam.latexexam.LatexExam method)@\spxentry{export\_tex\_solution()}\spxextra{pytexexam.latexexam.LatexExam method}}

\begin{fulllineitems}
\phantomsection\label{\detokenize{index:pytexexam.latexexam.LatexExam.export_tex_solution}}\pysiglinewithargsret{\sphinxbfcode{\sphinxupquote{export\_tex\_solution}}}{\emph{file\_name: str}}{}
Export a file containing detailed answers for each question in the exam

\end{fulllineitems}

\index{question\_theorem (pytexexam.latexexam.LatexExam attribute)@\spxentry{question\_theorem}\spxextra{pytexexam.latexexam.LatexExam attribute}}

\begin{fulllineitems}
\phantomsection\label{\detokenize{index:pytexexam.latexexam.LatexExam.question_theorem}}\pysigline{\sphinxbfcode{\sphinxupquote{question\_theorem}}\sphinxbfcode{\sphinxupquote{ = None}}}
The content of the beginning of each question will be printed

\end{fulllineitems}

\index{solution\_theorem (pytexexam.latexexam.LatexExam attribute)@\spxentry{solution\_theorem}\spxextra{pytexexam.latexexam.LatexExam attribute}}

\begin{fulllineitems}
\phantomsection\label{\detokenize{index:pytexexam.latexexam.LatexExam.solution_theorem}}\pysigline{\sphinxbfcode{\sphinxupquote{solution\_theorem}}\sphinxbfcode{\sphinxupquote{ = None}}}
The content of the beginning of each detailed answer will be printed

\end{fulllineitems}

\index{user\_preamble (pytexexam.latexexam.LatexExam attribute)@\spxentry{user\_preamble}\spxextra{pytexexam.latexexam.LatexExam attribute}}

\begin{fulllineitems}
\phantomsection\label{\detokenize{index:pytexexam.latexexam.LatexExam.user_preamble}}\pysigline{\sphinxbfcode{\sphinxupquote{user\_preamble}}\sphinxbfcode{\sphinxupquote{ = None}}}
Preamble of the latex file corresponds to the exam

\end{fulllineitems}


\end{fulllineitems}



\chapter{Pytexexam util class}
\label{\detokenize{index:module-pytexexam.latexexamutil}}\label{\detokenize{index:pytexexam-util-class}}\index{pytexexam.latexexamutil (module)@\spxentry{pytexexam.latexexamutil}\spxextra{module}}\index{ams\_math\_package() (in module pytexexam.latexexamutil)@\spxentry{ams\_math\_package()}\spxextra{in module pytexexam.latexexamutil}}

\begin{fulllineitems}
\phantomsection\label{\detokenize{index:pytexexam.latexexamutil.ams_math_package}}\pysiglinewithargsret{\sphinxcode{\sphinxupquote{pytexexam.latexexamutil.}}\sphinxbfcode{\sphinxupquote{ams\_math\_package}}}{}{{ $\rightarrow$ str}}
Returns the command lines needed to type math formula in latex

\end{fulllineitems}



\renewcommand{\indexname}{Python Module Index}
\begin{sphinxtheindex}
\let\bigletter\sphinxstyleindexlettergroup
\bigletter{p}
\item\relax\sphinxstyleindexentry{pytexexam.exam}\sphinxstyleindexpageref{index:\detokenize{module-pytexexam.exam}}
\item\relax\sphinxstyleindexentry{pytexexam.latexexam}\sphinxstyleindexpageref{index:\detokenize{module-pytexexam.latexexam}}
\item\relax\sphinxstyleindexentry{pytexexam.latexexamutil}\sphinxstyleindexpageref{index:\detokenize{module-pytexexam.latexexamutil}}
\item\relax\sphinxstyleindexentry{pytexexam.question}\sphinxstyleindexpageref{index:\detokenize{module-pytexexam.question}}
\end{sphinxtheindex}

\renewcommand{\indexname}{Index}
\printindex
\end{document}